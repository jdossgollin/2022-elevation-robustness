\documentclass{ar2rc}

\usepackage{siunitx}
\usepackage{apacite}
\usepackage{physics}

\usepackage[colorinlistoftodos]{todonotes}
\usepackage{color}
\newcommand{\jdg}[1]{\todo[color=red!30]{#1}}
\newcommand{\all}[1]{\todo[color=blue!30]{#1}}

% better lists
\usepackage{enumitem}
\setlist{nosep}

% some commands
\usepackage{xspace}
\makeatletter
\DeclareRobustCommand\onedot{\futurelet\@let@token\@onedot}
\def\@onedot{\ifx\@let@token.\else.\null\fi\xspace}
\def\eg{\emph{e.g}\onedot} \def\Eg{\emph{E.g}\onedot}
\def\ie{\emph{i.e}\onedot} \def\Ie{\emph{I.e}\onedot}
\def\etc{\emph{etc}\onedot} \def\vs{\emph{vs}\onedot}
\newcommand{\usd}[1]{\SI{#1}[\$]{}}

% ACRONYMS
\usepackage[acronym, nopostdot, nonumberlist, shortcuts, numberedsection, nogroupskip,]{glossaries}
\newacronym{bfe}{BFE}{base flood elevation}
\newacronym{cmip}{CMIP}{the Coupled Model Intercomparison Project}
\newacronym{dmdu}{DMDU}{decision making under deep uncertainty}
\newacronym{fema}{FEMA}{the Federal Emergency Management Agency}
\newacronym{gev}{GEV}{generalized extreme value}
\newacronym{hazus}{HAZUS}{Hazard U.S.}
\newacronym{iid}{IID}{independent and identically distributed}
\newacronym{ipcc}{IPCC}{International Panel on Climate Change}
\newacronym{msl}{MSL}{mean relative sea level}
\newacronym{noaa}{NOAA}{the National Oceanic and Atmospheric Administration}
\newacronym{pdf}{PDF}{probability density function}
\newacronym{rcp}{RCP}{representative concentration pathway}
\newacronym{rdm}{RDM}{robust decision making}
\newacronym{slr}{SLR}{sea level rise}
\newacronym{ssp}{SSP}{shared socio-economic pathway}
\newacronym[]{usace}{USACE}{United States Army Corps of Engineers}
\newacronym[]{usgs}{USGS}{United States Geological Survey}
\newacronym[\glslongpluralkey={states of the world}]{sow}{SOW}{state of the world}

% cross-refs
\usepackage{xr-hyper}
\makeatletter
\newcommand*{\addFileDependency}[1]{% argument=file name and extension
  \typeout{(#1)}
  \@addtofilelist{#1}
  \IfFileExists{#1}{}{\typeout{No file #1.}}
}
\makeatother
\newcommand*{\myexternaldocument}[1]{%
    \externaldocument{#1}%
    \addFileDependency{#1.tex}%
    \addFileDependency{#1.aux}%
}
\myexternaldocument{submission_03}
\myexternaldocument{supplemental}

\hypersetup{hidelinks}
\usepackage{cleveref}

\title{A subjective Bayesian framework for synthesizing deep uncertainties in climate risk management}
\author{James Doss-Gollin and Klaus Keller}
\journal{Earth's Future}

\begin{document}

\maketitle

We thank the editor and reviewer \# 2 for the thoughtful review and for the helpful suggestions that have helped us to substantially clarify and improve the manuscript.

\emph{All line numbers referenced in this document refer to the line numbers of the tracked changes document.}

\section*{Referee \#2}

\RC{
    This is my second review of the manuscript ``A subjective Bayesian framework for synthesizing deep uncertainties in climate risk management'' by Doss-Gollin and Keller.
    I thank the authors for a very thorough response to my previous comments.
    The manuscript has been greatly improved, and my previous concerns about the contribution and methods have been adequately addressed.
}

\AR{
    Thank you for your comments.
    We are glad that our comments addressed your concerns about the contribution and methods.
}

\RC{
    While the revised manuscript is much clearer, I still had trouble understanding the main steps and goals of the framework.
    It took me several close reads.
    I make some recommendations for improving this below so that the authors' important contribution will be clear to readers.
    Otherwise, I am happy to recommend the paper be accepted for publication.
}

\AR{
    We are very grateful for the thoughtful and helpful suggestions that have helped us to clarify the manuscript.
    We respond to each below.
}

\RC{
    As I understand it, the core of the method is that the SOW used in exploratory modeling can also be integrated into a single probabilistic description for classical decision analysis.
    This is done using a subjective probability distribution of future uncertainties, which is used to infer the implied weights over the SOW, making implicit assumptions about likelihoods of different SOW transparent to decision-makers.
    A main barrier the authors addressed this is that SOW are often generated from (or conditioned on) specific scenarios e.g. RCP 8.5 vs RCP 4.5, leading to the multiple pdf problem.
    Here, the authors integrate these multiple scenarios and associated SOW into a single distribution and corresponding weights.
}

\AR{
    This is a very articulate summary of our core goals and work.
}

\RC{
    If that summary is correct, I recommend that the authors summarize the aims and high-level approach of this contribution before diving into the methods details.
    At present, the framework is described in the paragraph in the intro on lines 110-122 and the paragraph opening the conceptual framework section (lines 151-155).
    However, I was not able to infer the main aims, steps, and contributions from these paragraphs.
    Revision of these paragraphs to tee up the big picture of the framework at the outset of the methods presentation would be very helpful to the reader.
}

\AR{
    We thank the reviewer for the helpful suggestion.
    We have rewritten the beginning of section 2 with these comments in mind, borrowing some language from the reviewer's summary provided above.
    See L152:
}

\begin{quote}
    In this section we introduce a conceptual framework and notation for decision analysis under deep uncertainty.
    \DIFdelbegin \DIFdel{We then apply this framework to understand exploratory modeling (\mbox{%DIFAUXCMD
            \cref{sec:analysis-explore}}\hskip0pt%DIFAUXCMD
        )  and scenario-conditional probabilistic analysis (\mbox{%DIFAUXCMD
            \cref{sec:analysis-condition}}\hskip0pt%DIFAUXCMD
        ).
        In \mbox{%DIFAUXCMD
            \cref{sec:analysis-synthesize} }\hskip0pt%DIFAUXCMD
        we discuss why synthesizing across multiple scenarios is necessary.
        In \mbox{%DIFAUXCMD
            \cref{sec:reweighting} }\hskip0pt%DIFAUXCMD
        we provide a formal method for synthesizing across scenarios.
    }\DIFdelend \DIFaddbegin \DIFadd{Many bottom-up exploratory modeling frameworks used in climate risk management and related fields use a system model ($f$ in \mbox{%DIFAUXCMD
            \cref{fig:flowchart}}\hskip0pt%DIFAUXCMD
        )  to characterize the system's response to a wide range of plausible futures, often called \mbox{%DIFAUXCMD
            \glspl{sow} }\hskip0pt%DIFAUXCMD
        ($\vb{s}$ in \mbox{%DIFAUXCMD
            \cref{fig:flowchart}}\hskip0pt%DIFAUXCMD
        ).
        This analysis is often used to explore vulnerabilities and build knowledge about the system \mbox{%DIFAUXCMD
            \cite{bankes:1993}}\hskip0pt%DIFAUXCMD
        , and in general exploratory modeling frameworks aim to avoid making explicit judgments about the relative likelihood of different futures.
    }\DIFaddend

    \DIFdelbegin \DIFdel{It is common in climate risk management to work with an ensemble of trajectories of climate, economic, and other key variables, which we refer to as \mbox{%DIFAUXCMD
            \glspl{sow}}\hskip0pt%DIFAUXCMD
        .
        Often these \mbox{%DIFAUXCMD
            \glspl{sow} }\hskip0pt%DIFAUXCMD
        are computationally expensive to generate, and thus it is important to use them efficiently.
        Following \mbox{%DIFAUXCMD
            \cref{fig:flowchart}}\hskip0pt%DIFAUXCMD
        , we consider using $J$ \mbox{%DIFAUXCMD
            \glspl{sow} }\hskip0pt%DIFAUXCMD
        , $\vb{s} = \qty{s_1, s_2, \ldots, s_J}$, to evaluate $I$ candidate decisions, $\vb{x} = \qty{x_1, x_2, \ldots, x_I}$.
        For each \mbox{%DIFAUXCMD
            \gls{sow} }\hskip0pt%DIFAUXCMD
        $s_j \in \vb{s}$ and decision $x_i \in \vb{x}$ we use a system model $f$ (comprised of multiple components) to calculate a set of metrics describing the performance of decision $x_i$ on \mbox{%DIFAUXCMD
            \gls{sow} }\hskip0pt%DIFAUXCMD
        $s_j$, which we denote $u_{ij} = f(x_i, s_j)$.
        While we assume for simplicity that }\DIFdelend \DIFaddbegin \DIFadd{However, as discussed in \mbox{%DIFAUXCMD
            \cref{sec:introduction}}\hskip0pt%DIFAUXCMD
        , estimates of trade-offs between desired performance metrics (}\eg\DIFadd{, cost and reliability) depend on probabilistic models of uncertainty.
        In this paper we present a method for integrating the \mbox{%DIFAUXCMD
            \glspl{sow} }\hskip0pt%DIFAUXCMD
        used in exploratory modeling into a formal decision analytic framework using a subjective probability distribution over the space of possible futures, which is used to infer implicit weights over the \mbox{%DIFAUXCMD
            \glspl{sow}}\hskip0pt%DIFAUXCMD
        .
        This approach is particularly suited for problems where the \mbox{%DIFAUXCMD
            \glspl{sow} }\hskip0pt%DIFAUXCMD
        are generated from or conditioned on specific scenarios (}\eg\DIFadd{, \mbox{%DIFAUXCMD
            \gls{rcp} }\hskip0pt%DIFAUXCMD
        scenarios) or where there are multiple models of the underlying processes (}\eg\DIFadd{, multiple models of }\DIFaddend the \DIFdelbegin \DIFdel{decision space is known and finite, this approach could be coupled to a policy search model that proposes candidate decisions}\DIFdelend \DIFaddbegin \DIFadd{response of local sea levels to global temperature), which can lead to the ``multiple \mbox{%DIFAUXCMD
            \gls{pdf} }\hskip0pt%DIFAUXCMD
        problem'' (discussed in \mbox{%DIFAUXCMD
            \cref{sec:analysis-condition}}\hskip0pt%DIFAUXCMD
        ).
        A motivating advantage is that it makes assumptions about the likelihoods of different \mbox{%DIFAUXCMD
            \glspl{sow} }\hskip0pt%DIFAUXCMD
        transparent to decision-makers}\DIFaddend .
\end{quote}

\RC{
    One thing that may help is clarification around terminology:
}

\RC{
    First, the authors use the phrase ``synthesize deep uncertainties'' throughout.
    I find this vague and unhelpful.
    I think what you mean by this is integrating scenario-based deep uncertainties into a single probabilistic representation.
    I suggest you either clearly define what you mean by this phrase when you introduce it and/or use more specific descriptive language.
}

\AR{
    We appreciate this clarifying suggestion.
    To address this comment we have more clearly defined what is meant by ``synthesizing'' deep uncertainties (see paragraph beginning L110):
}

\begin{quote}
    In this paper we offer a conceptual step towards bridging this divide by presenting a framework that is designed to combine the strengths of both approaches.
    In the first step, exploratory or bottom-up modeling is used to build insight and identify potential system vulnerabilities \cite{moallemi_exploratory:2020,bankes:1993,brown_decisionscaling:2012}.
    In the second step, we \DIFdelbegin \DIFdel{synthesize the results of exploratory modeling to }\DIFdelend \DIFaddbegin \DIFadd{integrate exploratory ensembles of deep uncertainties into a single probabilistic representation (we refer to this as ``synthesizing'' deep uncertainties) to }\DIFaddend formally estimate performance metrics and trade-offs using \DIFdelbegin \DIFdel{probabilistic models for uncertainty}\DIFdelend \DIFaddbegin \DIFadd{subjective probability distributions}\DIFaddend .
    Drawing from the literature on building predictive models when all models are wrong \cite{box_sciencestatistics:1976,gelman_philosophy:2013,Piironen:2017eh}, we interpret these probability distributions not as statements of fact, but rather as a self-consistent framework for reasoning about how different assumptions lead to different inferences.
    An advantage of our approach is that it facilitates computationally efficient analysis of how alternative probabilistic models would affect estimated performance metrics and trade-offs.
\end{quote}

\RC{
    Second, while the specific application helped me understand how you differentiate between SOW and ``scenarios'', this was confusing in Section 2.
    Your definition of SOW on lines 156-158 should be clarified.
    I suggest you define what you mean by ``scenario'' here and how SOW and scenarios are related.
    In lines 186-188, you introduce ``a particular scenario $M_k$'' without defining scenario.
    It wasn't obvious to me at first read that the emissions pathway is the scenario.
}

\AR{
    Thanks for this helpful comment.
    We hope that the text provided above provides some clarification.
    We provide a further clarification of the difference between a \gls{sow} and scenario in \cref{sec:analysis-condition}.
    See L202:
}

\begin{quote}
    One way to interpret an ensemble of \glspl{sow} is as {iid} draws from some probabilistic data generating process.
    This commonly arises when a single deep uncertainty (\eg, an emissions pathway) is used as an input for a stochastic model.
    \DIFaddbegin \DIFadd{To clarify language, we draw a distinction between a \mbox{%DIFAUXCMD
            \gls{sow}}\hskip0pt%DIFAUXCMD
        , which is a single realization of a possible future, and a }\emph{\DIFadd{scenario}}\DIFadd{, which is a model that can be used to generate \mbox{%DIFAUXCMD
            \glspl{sow}}\hskip0pt%DIFAUXCMD
        .
        These terms are defined relative to the problem at hand; for example, a global emissions trajectory may be a \mbox{%DIFAUXCMD
            \gls{sow} }\hskip0pt%DIFAUXCMD
        from an integrated assessment model, but when it is used as a deterministic input to an ice sheet model we would describe it as a scenario.
    }

    \DIFaddend We illustrate this \DIFdelbegin \DIFdel{concept }\DIFdelend \DIFaddbegin \DIFadd{distinction }\DIFaddend in boxes (d) and (e) of \cref{fig:flowchart}, denoting the particular scenario $M_k$.
    \DIFdelbegin \DIFdel{By treating the \mbox{%DIFAUXCMD
            \glspl{sow} }\hskip0pt%DIFAUXCMD
        as \mbox{%DIFAUXCMD
            \gls{iid} }\hskip0pt%DIFAUXCMD
        draws }\DIFdelend \DIFaddbegin \DIFadd{We assume that each scenario is probabilistic, that is that \mbox{%DIFAUXCMD
            \glspl{sow} }\hskip0pt%DIFAUXCMD
        are drawn \mbox{%DIFAUXCMD
            \gls{iid} }\hskip0pt%DIFAUXCMD
    }\DIFaddend from $M_k$, the set of outcomes $u_{i, j}$ can be interpreted as \gls{iid} draws from the conditional distribution over outcomes, $p(u | x_i, M_k)$.
    This ``scenario-conditional'' probabilistic interpretation of \glspl{sow} allows for fully probabilistic quantification of uncertainty and optimization, conditional on a particular scenario.
    For example,\ldots
\end{quote}

\RC{
    Relatedly, I understand the scenario-conditional interpretation in the example with RCP pathways in lines 191-198, but I don't follow this interpretation in 199-219.
    I agree sampling SOW from fixed ranges implies a uniform probability distribution over the SOW, but what is the ``scenario'' here?
}

\AR{
    Thanks for pointing out our unclear reasoning.
    We have made the following changes to clarify how the theoretical lens of ``scenario-conditional analysis'' can be used to explore these approaches.
    See L222:
}

\begin{quote}
    We can also \DIFdelbegin \DIFdel{characterize \mbox{%DIFAUXCMD
            \gls{dmdu} }\hskip0pt%DIFAUXCMD
        methods that sample a set of parameters from fixed ranges as scenario-conditional analyses}\DIFdelend \DIFaddbegin \DIFadd{apply this theoretical lens to examine the approach, common in \mbox{%DIFAUXCMD
            \gls{dmdu} }\hskip0pt%DIFAUXCMD
        applications, of sampling parameters from a set of fixed ranges.
        The }\emph{\DIFadd{scenario}} \DIFadd{in this case is thus the choice of bounds on the parameters; it is consistent with our above definition of a scenario because it provides a probabilistic model that can be used to sample \mbox{%DIFAUXCMD
            \glspl{sow}}\hskip0pt%DIFAUXCMD
    }\DIFaddend .
    For example, \ldots
\end{quote}

\RC{
    Third, the conceptual definition of $p_\mathrm{belief}$ and its relation to the weights was unclear to me when you introduce them lines 242 - 252.
    You say that estimating the weights require a probability distribution.
    But it wasn't clear that what you mean is that you choose a probability distribution, reflecting subjective belief about the SOW, and then you infer what weights over the SOW are consistent with that subjective belief.
    That is very powerful but the lack of conceptual definitions and vague language at the beginning of the section like ``we provide a formal method for synthesizing across multiple scenarios to inform climate risk management'' made it hard to understand this.
}

\AR{
    Thanks for pointing out this opportunity to improve the clarity of our text.
    We have clarified this paragraph to emphasize the choice of subjective probability distribution.
    See L274:
}

\begin{quote}
    \ldots
    Then, we partition the parameter space into a region corresponding to each \gls{sow} and integrating the probability $p(\psi)$ over each region.

    \DIFdelbegin \DIFdel{Doing so requires a probabilistic model }\DIFdelend \DIFaddbegin \DIFadd{Implementing this approach requires choosing a probability distribution }\DIFaddend for this low-dimensional representation of the \glspl{sow},  $p(\psi)$\DIFaddbegin \DIFadd{, reflecting subjective belief about the \mbox{%DIFAUXCMD
            \glspl{sow}}\hskip0pt%DIFAUXCMD
    }\DIFaddend .
    We denote this \DIFdelbegin \DIFdel{model $p_\mathrm{belief}$ }\DIFdelend \DIFaddbegin \DIFadd{$p_\mathrm{belief}(\psi)$ }\DIFaddend to emphasize that it represents a subjective belief about the \glspl{sow}, rather than an objectively verifiable choice.
    In general we do not\ldots
\end{quote}

\AR{
    In addition, we have reworded the vague language referenced by the reviewer at the start of the subsection (L263):
}

\begin{quote}
    In this section we provide a formal method for \DIFdelbegin \DIFdel{synthesizing across multiple scenarios to inform climate risk management}\DIFdelend \DIFaddbegin \DIFadd{integrating exploratory ensembles of deep uncertainties into a single probabilistic representation}\DIFaddend .
    Our objective is to\ldots
\end{quote}

\RC{
    Finally, in your response to my previous comments, you said that you removed references to $p_\mathrm{belief}$ as priors.
    This is very helpful - that clarifies you are not performing Bayeisan analysis, and certainly a probability distribution can still represent subjective belief even if it is not a prior in a Bayesian framework.
    However, the text still contains many references to $p_\mathrm{belief}$ as a prior, especially in the results section, \eg figure 8 and surrounding text.
}

\AR{
    We thank the reviewer for drawing this to our attention.
    We have removed all remaining references to $p_\mathrm{belief}$ as a prior.
    This includes changing the the subplot title for figure 8(a).
}

% ugly hack to hide month in APACite
\renewcommand{\APACrefYearMonthDay}[3]{\APACrefYear{#1}}
\bibliographystyle{apacite}
\bibliography{library-bibtex}

\end{document}

\begin{block}{key message}
    \vspace*{-1em}
    \begin{framed}
        \begin{figure}
            \centering
            \includegraphics[width=0.48\textwidth]{lsl_priors.pdf}~
            \includegraphics[width=0.48\textwidth]{tradeoffs_height_totalcost_byprior.pdf}~
            \caption{
                The prior model averaging approach (\cref{fig:flowchart}), uses a probability distribution representing subjective belief to average insights across each PDF available.
                Alternative priors can provide robustness checks.
            }
        \end{figure}
    \end{framed}
    \begin{itemize}
        \item Natural extensions to Bayesian updating
        \item Belief $p(s)$ can't be ``right''  \cite{gelman_workflow:2020,gelman_philosophy:2013} or ``neutral'' \cite{quinn_exploratory:2020}
        \item Instead, we can ask \textbf{what assumptions are different priors consistent with?}
    \end{itemize}
    \begin{framed}
        \begin{figure}
            \centering
            \includegraphics[width=\textwidth]{inference_weights.pdf}
            \caption{
                The weights assigned to each PDF (from \cref{fig:boxplots}) under each prior considered.
            }
        \end{figure}
    \end{framed}
    ``Pessimistic'' (``Optimistic'') prior heavily weights \gls{rcp} 8.5 (2.6), which is unlikely given current policies \cite{hausfather_scenarios:2020}.
\end{block}